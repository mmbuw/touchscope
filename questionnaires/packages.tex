% ---------------------------------------------------------------------
% Packages
% ---------------------------------------------------------------------

\usepackage[utf8]{inputenc}
\usepackage{amsmath}
\usepackage{amssymb}
\usepackage[ngerman,english]{babel}
\usepackage[usenames, dvipsnames]{color}
\usepackage{graphicx}
\usepackage{epsfig}
\usepackage{latexsym}
\usepackage{listings}
\usepackage{times}
\usepackage{url}
\usepackage{xspace}
\usepackage[normalem]{ulem}
\usepackage{textcomp}
\usepackage{subfiles}
\usepackage{float}
\usepackage{multirow}
\usepackage{pifont}
\usepackage{csquotes}
\usepackage{cancel}
\usepackage{titlesec}
\usepackage{chngcntr}

\usepackage[
 pdftex,
 pdfauthor={Matthew Heinz},
 pdftitle={Learning Task},
 pdfsubject={Touch Scope Learning Task},
 pdfproducer={Latex with hyperref},
 pdfcreator=pdflatex,
 hidelinks
]{hyperref}

\usepackage[a4paper, margin=1in]{geometry}


% ---------------------------------------------------------------------
% Commands
% ---------------------------------------------------------------------
\newcommand{\cmark}{\color{OliveGreen}\ding{51}}
\newcommand{\xmark}{\color{BrickRed}\ding{55}}

% to add subsubsubsection
\titleclass{\subsubsubsection}{straight}[\subsection]
\newcounter{subsubsubsection}[subsubsection]
\renewcommand\thesubsubsubsection{\thesubsubsection.\arabic{subsubsubsection}}
\titleformat{\subsubsubsection}
  {\normalfont\normalsize\bfseries}{\thesubsubsubsection}{1em}{}
\titlespacing*{\subsubsubsection}
{0pt}{3.25ex plus 1ex minus .2ex}{1.5ex plus .2ex}

\makeatletter
\def\toclevel@subsubsubsection{4}
\def\l@subsubsubsection{\@dottedtocline{4}{7em}{4em}}
\makeatother

\setcounter{secnumdepth}{4}
% end subsubsubsection

\setcounter{tocdepth}{2}
\counterwithin{table}{section}
\counterwithin{figure}{section}
\AtBeginDocument{\counterwithin{lstlisting}{section}}

% ---------------------------------------------------------------------
% Display Source Code Settings
% ---------------------------------------------------------------------

\lstset{ %
  backgroundcolor=\color{white},   % choose the background color; you must add \usepackage{color} or \usepackage{xcolor}
  basicstyle=\footnotesize\ttfamily, % the size of the fonts that are used for the code
  breakatwhitespace=false,         % sets if automatic breaks should only happen at whitespace
  breaklines=true,                 % sets automatic line breaking
  captionpos=b,                    % sets the caption-position to bottom
  commentstyle=\color{OliveGreen},  	   % comment style
  deletekeywords={...},            % if you want to delete keywords from the given language
  escapeinside={\%*}{*)},          % if you want to add LaTeX within your code
  extendedchars=true,              % lets you use non-ASCII characters; for 8-bits encodings only, does not work with UTF-8
  frame=single,                    % adds a frame around the code
  keepspaces=true,                 % keeps spaces in text, useful for keeping indentation of code (possibly needs columns=flexible)
  keywordstyle=\color{blue},       % keyword style
  language=Java,                   % the language of the code
  morekeywords={*,...},            % if you want to add more keywords to the set
  numbers=left,                    % where to put the line-numbers; possible values are (none, left, right)
  numbersep=5pt,                   % how far the line-numbers are from the code
  numberstyle=\tiny\color{OliveGreen},  % the style that is used for the line-numbers
  rulecolor=\color{black},         % if not set, the frame-color may be changed on line-breaks within not-black text (e.g. comments (green here))
  showspaces=false,                % show spaces everywhere adding particular underscores; it overrides 'showstringspaces'
  showstringspaces=false,          % underline spaces within strings only
  showtabs=true,                   % show tabs within strings adding particular underscores
  stepnumber=1,                    % the step between two line-numbers. If it's 1, each line will be numbered
  stringstyle=\color{blue},        % string literal style
  tabsize=2,                       % sets default tabsize to 2 spaces
  title=\lstname                   % show the filename of files included with \lstinputlisting; also try caption instead of title
}


% ---------------------------------------------------------------------
% Paths
% ---------------------------------------------------------------------

\graphicspath{{./images/}}
